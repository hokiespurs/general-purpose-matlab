\documentclass{article}
\usepackage[margin=1in]{geometry} 
\usepackage{amsmath}
\usepackage[T1]{fontenc}          % change font encoding to T1
\usepackage{lmodern}  %better for visual on screen
\usepackage{graphicx}
\usepackage{float}
\usepackage{enumitem}
\usepackage{mathtools}
\usepackage{booktabs}
\usepackage{dirtree}

\makeatletter
\renewcommand*\env@matrix[1][\arraystretch]{%
	\edef\arraystretch{#1}%
	\hskip -\arraycolsep
	\let\@ifnextchar\new@ifnextchar
	\array{*\c@MaxMatrixCols c}}
\makeatother

% Used for adding Matlab Algorithms
\RequirePackage{listings}
\RequirePackage[framed,numbered]{matlab-prettifier}

\DeclarePairedDelimiter\abs{\lvert}{\rvert}%

\begin{document}
	\section{plothelpers.m}
	\subsection{Motivation/Concept}
	These functions are used so that it is easier to make plots.  The following functions are included:
	\subsubsection{axgrid.m}
	This function helps generate a grid of subplots with more explicit control on their spacing.
	\subsubsection{bigcolorbar.m}
	This function helps generate a standalone colorbar wherever you want.
	\subsubsection{bigcolorbarax.m}
	This function helps generate a standalone colorbar which aligns with a set axes handles input to the function.
	\subsubsection{bigtitle.m}
	This function helps generate a standalone title anywhere on the screen.
	\subsubsection{bigtitleax.m}
	This function helps generate a standalone title which aligns itself centered above a given set of axes handles.
	\subsubsection{fixfigstring.m}
	This function replaces an underscore in a string with so that it prints appropriately.
	\subsection{linkax.m}
	This function allows you to link any(unlike linkaxes) axes property between different axes.
	\subsubsection{maxpos.m}
	This function returns the position bounds of an array of axes. It is used by \textit{bigcolorbarax.m} and \textit{bigtitleax.m}.
	\clearpage
	\subsection*{Example Usage (\textit{examplePlotHelpers.m})}
	This example demonstrates many of the functions in plotHelpers.
	\lstinputlisting[
	label      = {alg:lsr2},
	style      = Matlab-editor,
	basicstyle = \mlttfamily,
	firstline  = 1,
	]{../exampleplothelpers.m}
	
	\begin{figure}[H]
		\centering
		\includegraphics[height = 4in]{exampleplothelpers}
	\end{figure}
\end{document}